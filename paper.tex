%%
%% This is file `sample-sigconf.tex',
%% generated with the docstrip utility.
%%
%% The original source files were:
%%
%% samples.dtx  (with options: `sigconf')
%%
%% IMPORTANT NOTICE:
%%
%% For the copyright see the source file.
%%
%% Any modified versions of this file must be renamed
%% with new filenames distinct from sample-sigconf.tex.
%%
%% For distribution of the original source see the terms
%% for copying and modification in the file samples.dtx.
%%
%% This generated file may be distributed as long as the
%% original source files, as listed above, are part of the
%% same distribution. (The sources need not necessarily be
%% in the same archive or directory.)
%%
%% The first command in your LaTeX source must be the \documentclass command.
\documentclass[sigconf,10pt]{acmart}

\settopmatter{printfolios=true}

\usepackage{hyperref}


\providecommand{\tightlist}{%
  \setlength{\itemsep}{0pt}\setlength{\parskip}{0pt}}

%%
%% \BibTeX command to typeset BibTeX logo in the docs
\AtBeginDocument{%
  \providecommand\BibTeX{{%
    \normalfont B\kern-0.5em{\scshape i\kern-0.25em b}\kern-0.8em\TeX}}}

%% Rights management information.  This information is sent to you
%% when you complete the rights form.  These commands have SAMPLE
%% values in them; it is your responsibility as an author to replace
%% the commands and values with those provided to you when you
%% complete the rights form.


%%
%% Submission ID.
%% Use this when submitting an article to a sponsored event. You'll
%% receive a unique submission ID from the organizers
%% of the event, and this ID should be used as the parameter to this command.
%%\acmSubmissionID{123-A56-BU3}

%%
%% The majority of ACM publications use numbered citations and
%% references.  The command \citestyle{authoryear} switches to the
%% "author year" style.
%%
%% If you are preparing content for an event
%% sponsored by ACM SIGGRAPH, you must use the "author year" style of
%% citations and references.
%% Uncommenting
%% the next command will enable that style.
%%\citestyle{acmauthoryear}

%%
%% end of the preamble, start of the body of the document source.
\begin{document}

%%
%% The "title" command has an optional parameter,
%% allowing the author to define a "short title" to be used in page headers.
\title{A Unified Model For Web Scraping \& Customization}

%%
%% The "author" command and its associated commands are used to define
%% the authors and their affiliations.
%% Of note is the shared affiliation of the first two authors, and the
%% "authornote" and "authornotemark" commands
%% used to denote shared contribution to the research.

\author{Kapaya Katongo}
\affiliation{%
  \institution{MIT CSAIL}
  \city{Cambridge, MA}
  \country{USA}
}
\email{kkatongo@mit.edu}

\author{Geoffrey Litt}
\affiliation{%
  \institution{MIT CSAIL}
  \city{Cambridge, MA}
  \country{USA}
}
\email{glitt@mit.edu}

\author{Kathryn Jin}
\affiliation{%
  \institution{MIT CSAIL}
  \city{Cambridge, MA}
  \country{USA}
}
\email{kjin@mit.edu}

\author{Daniel Jackson}
\affiliation{%
  \institution{MIT CSAIL}
  \city{Cambridge, MA}
  \country{USA}
}
\email{dnj@csail.mit.edu}

%%
%% By default, the full list of authors will be used in the page
%% headers. Often, this list is too long, and will overlap
%% other information printed in the page headers. This command allows
%% the author to define a more concise list
%% of authors' names for this purpose.
% \renewcommand{\shortauthors}{Trovato and Tobin, et al.}

%%
%% The abstract is a short summary of the work to be presented in the
%% article.
\begin{abstract}
  Websites are malleable: users can run code in the browser to customize
  them. However, this malleability is typically only accessible to
  programmers with knowledge of HTML and Javascript.
\end{abstract}

%%
%% The code below is generated by the tool at http://dl.acm.org/ccs.cfm.
%% Please copy and paste the code instead of the example below.
%%
%% From HERE
\begin{CCSXML}
<ccs2012>
<concept>
<concept_id>10011007.10011006.10011066.10011069</concept_id>
<concept_desc>Software and its engineering~Integrated and visual development environments</concept_desc>
<concept_significance>500</concept_significance>
</concept>
</ccs2012>
\end{CCSXML}

\ccsdesc[500]{Software and its engineering~Integrated and visual development environments}
% To HERE

%%
%% Keywords. The author(s) should pick words that accurately describe
%% the work being presented. Separate the keywords with commas.
\keywords{end-user programming, software customization, web scraping, programming by example}

%% A "teaser" image appears between the author and affiliation
%% information and the body of the document, and typically spans the
%% page.
%\begin{teaserfigure}
%  \includegraphics[width=\textwidth]{sampleteaser}
%  \caption{Seattle Mariners at Spring Training, 2010.}
%  \Description{Enjoying the baseball game from the third-base
%  seats. Ichiro Suzuki preparing to bat.}
%  \label{fig:teaser}
%\end{teaserfigure}

%%
%% This command processes the author and affiliation and title
%% information and builds the first part of the formatted document.
\maketitle

\hypertarget{sec:introduction}{%
\section{Introduction}\label{sec:introduction}}

Many websites on the internet do not meet the exact needs of all of
their users. End-user web customization systems like Thresher, Sifter
and Vegemite help users to tweak and adapt websites to fit their unique
requirements, ranging from reorganizing or annotating content on the
website to automating common tasks. Millions of people also use tools
like Greasemonkey and Tampermonkey to install browser userscripts,
snippets of Javascript code which customize the behavior of websites.

A new approach known as data-driven web customization promises to enable
web customization without traditional programming through direct
manipulation of a spreadsheet-like table, right within the context of a
browser. In this paradigm, a table is added to a website that contains
its underlying structured data and is bidirectionally synchronized with
it. Changes to the table, including sorting, filtering, adding
annotations and running computations in a spreadsheet-like formula
language are propagated to the website thereby customizing it.

While end-user friendly, data-driven customization suffers from a divide
between generating the table and using it to perform customizations: the
table is generated through web scraping code written by programmers but
used via direct manipulation by end-users. This divide limits the agency
of end-users because they rely on programmers to write the required web
scraping code before being able to customize. In prior work, we
harnessed programming-by-demonstration to empower end-users to achieve
the task of web scraping, without writing code, within the context of
the table used for customizations. However, this reduced the
expressiveness of web scraping for programmers and both hid the
underlying program synthesized to perform the web scraping and prevented
modifications to be made to it.

The next task was therefore to achieve the seemingly opposing goals of
empowering end-users to perform web scraping within the context of the
table without disempowering programmers from being able to utilize the
expressiveness of programming and view and modify the web scraping
program. To solve this, we present a unified model for web scraping and
customization, with the aforementioned table formula language acting as
a bridge between the two. Demonstrations are used to empower end-users
to achieve the task of web scraping, with the synthesized web scraping
program presented as a formula in the table in a manner that enables
programmers to mitigate the lack of expressiveness in demonstrations by
being able to view and modify the synthesized program.

This model for web scraping and customization builds on several key
ideas and design goals we discuss in Section X. To test its viability,
we implement it as an extension of Wildcard, a browser extension that
implements data-driven web customization. Our contributions are:

\begin{itemize}
\tightlist
\item
  A unified model for web scraping and customization that enablers
  end-users to perform the task of web scraping and programmers to view
  and modify the web scraping program synthesized from a demonstration,
  both within the context of the table used for customization
\item
  A novel combination of design goals which make our unified model for
  web scraping and customization interactive and seamless for both
  end-users and programmers
\item
  An example gallery of websites that can be customized via our unified
  model for web scraping and customization and a preliminary user study
  providing some qualitative results of using it
\end{itemize}

\hypertarget{sec:demos}{%
\section{Motivating Example}\label{sec:demos}}

\hypertarget{sec:implementation}{%
\section{System Implementation}\label{sec:implementation}}

\hypertarget{wrapper-induction-algorithm}{%
\subsection{Wrapper Induction
Algorithm}\label{wrapper-induction-algorithm}}

\hypertarget{css-selector-synthesis}{%
\subsection{CSS Selector Synthesis}\label{css-selector-synthesis}}

\hypertarget{live-programming}{%
\subsection{Live Programming}\label{live-programming}}

\hypertarget{limitations}{%
\subsection{Limitations}\label{limitations}}

\hypertarget{wrapper-induction-algorithm-1}{%
\subsubsection{Wrapper Induction
Algorithm}\label{wrapper-induction-algorithm-1}}

\hypertarget{css-selector-synthesis-1}{%
\subsubsection{CSS Selector Synthesis}\label{css-selector-synthesis-1}}

\hypertarget{live-programming-1}{%
\subsubsection{Live Programming}\label{live-programming-1}}

\hypertarget{sec:design-principles}{%
\section{Design Principles}\label{sec:design-principles}}

\hypertarget{unified-environment-user-model}{%
\subsection{Unified Environment \& User
Model}\label{unified-environment-user-model}}

In the first iteration of data-driven customization, web scraping and
customization were divided: web scraping was done by programmers in an
Integrated Development Environment (IDE) while customizing (via direct
manipulation and formulas) was done in the browser. This type of divide
between tasks can be seen in other domains:

In data science, workflows revolve between cleaning and using data which
often happen in different environments (e.g.~data wrangling tools and
live notebooks). Wrex \citep{drosos2020}, an end-user
programming-by-example system for data wrangling, reported that
``although data scientists were aware of and appreciated the
productivity benefits of existing data wrangling tools, having to leave
their native notebook environment to perform wrangling limited the
usefulness of these tools.'' This was a major reason Wrex was developed
as an add-on to Jupyter notebooks, the environment in which data
scientists use their data. In web scraping, users have to switch from
the environment in which they are using the scraped data (database,
spreadsheet etc) to the environment in which the data is scraped if they
need more or need to fix omissions. This can be seen in tools like
Rousillon, FlashExtract, import.io, dexi.io, Octoparse and ParseHub.

Based on this, the second iteration of data-driven customization enabled
web scraping (via demonstrations) and customization to both be performed
in a uniform environment via the customization table in the browser.
This relates to the idea of ``in-place toolchains'' for end-user
programming systems: users should be able to program using familiar
tools (spreadsheet table) in the context where they already use their
software (browsers).

In spite of this uniform environment, web scraping and customization
were still divided: web scraping had to be performed prior to
customization in a separate phase. Early user tests revealed that this
discontinuity between the two phases was a source of confusion.
Vegemite, a system for end-user programming of mashups, reported similar
findings from its user study in which participants thought that ``it was
confusing to use one technique to create the initial table, and another
technique to add information to a new column.''

Armed with this, the iteration of data-driven customization we present
has gone beyond providing a uniform environment for web scraping and
customization to providing a unified user model for web scraping and
customization. Both web scraping and customization are performed in the
same, single phase, with users being able to seamlessly interleave the
two as desired. A user can start out by demonstrating to populate a
column in the table, proceed to populate the next column with the
results of a formula and then either switch back to demonstrating to
populate additional columns or continue populating columns with the
results of formula operations on the scraped data.

\hypertarget{functional-reactive-programming}{%
\subsection{Functional Reactive
Programming}\label{functional-reactive-programming}}

In general terms, functional reactive programming (FRP) is the
combination of functional and reactive programming: it is functional in
that it uses functions to define programs that manipulate data and
reactive in that it defines data flows through which changes in data are
propagated.

FRP has seen wide adoption in end-user programming through
implementations such as spreadsheet formula languages (Microsoft Excel
\& Google Sheets) and formula languages for low-code programming
environments (Microsoft Power Fx, Google AppSheets, Glide, Coda \&
Gneiss).

Because of this, data-driven web customization already provides
functional reactive programming via a spreadsheet-lie formula language
aimed at increasing the expressiveness of customizations. The language
provides formulas to encapsulate logic, perform operations on strings,
call browser APIs and even invoke web APIs. As per the FRP paradigm,
users only have to think in terms of manipulating the data in the table
without having to worry about traditional programming concepts such as
variables and data flow.

Our unified model for web scraping and customizations extends this
formula language to mitigate the limitations of programming by
demonstration. Demonstrations are represented as formulas containing the
corresponding, synthesized web scraping code. As with other formulas in
the language, web scraping formulas can be modified (or authored from
scratch) and run to achieve more expressive or robust web scraping.

Aside from the programmatic benefits of representing demonstrations as
formulas, there is also a key usability benefit. Mayer et al report that
``a key impedance in adoption of PBE systems is the lack of user
confidence in the correctness of the program that was synthesized by the
system.'' They describe how even though FlashFill, a
programming-by-example tool for string manipulation in Excel, received
many positive reviews from popular media sources, a prominent Excel user
expressed caution because of the lack of insight into what the
synthesized program is actually doing. Similar sentiments were reported
by the authors of Wrex who interviewed data scientists that ``were
reluctant to use data wrangling tools that transformed their data
through `black boxes.'\,''

By representing the demonstrations as formulas, our system provides
insight into what the synthesized code is doing. In future work, we aim
to emulate Mayer et el's work by providing a means for users to choose
between possible web scraping programs and pick the one that best
achieves their desired goal.

\hypertarget{mixed-initiative-interaction}{%
\subsection{Mixed-Initiative
Interaction}\label{mixed-initiative-interaction}}

Early ideas about mixed-initiative interactions can be traced to the
work of Eric Horvitz in which he advocates for ``designs that take
advantage of the power of direct manipulation and potentially valuable
automated reasoning.''

Our unified model for web scraping and customization offers
mixed-initiative interaction by presenting the result of web scraping by
demonstration as a formula. This is advantageous as it not only allows
users to delegate automation (via synthesis of web scraping programs) to
the system via demonstrations but also keeps the interaction loop open
by allowing users to view and modify the output of the demonstration.If
desired, users can also achieve web scraping by manually authoring web
scraping formulas, switching to generating them via demonstrations at
anytime in the process in a seamless and fluid manner.

This type of mixed-initiative interaction can be seen in other
programming-by-example systems: - Wrex takes examples of data transforms
and generates readable and editable Python code. This was motivated by
their formative study in which participants emphasized the need for
programming-by-example systems to ``produce code as an inspectable and
modifiable artifact''

\begin{itemize}
\item
  Small-Step Live Programming By Example presents a paradigm in which
  programming-by-example is used to synthesize small parts of a user
  authored program instead of delegating construction of entire program
\item
  Pileg et al outline how programming-by-example is not enough to
  differentiate all the possible programs and present an interaction
  model in which users not only provide feedback about the expected
  output of the program but also the program itself
\item
  Sketch-N-Sketch integrates direct manipulation and programming for the
  creation of Scalable Vector Graphics (SVG). Users can start out by
  creating a shape via programming and then switch to modifying its size
  or shape via direct manipulation which updates the underlying program
  to reflect the changes. Their central theme is that users do not have
  to choose between direct manipulation and programmatic systems
\end{itemize}

\hypertarget{live-programming-2}{%
\subsection{Live Programming}\label{live-programming-2}}

\hypertarget{sec:evaluation}{%
\section{Evaulation}\label{sec:evaluation}}

\hypertarget{example-gallary}{%
\subsection{Example Gallary}\label{example-gallary}}

\hypertarget{user-study}{%
\subsection{User Study}\label{user-study}}

\hypertarget{cognitive-dimensions-of-notation}{%
\subsection{Cognitive Dimensions Of
Notation}\label{cognitive-dimensions-of-notation}}

\hypertarget{sec:related-work}{%
\section{Related Work}\label{sec:related-work}}

\hypertarget{sec:conclusion}{%
\section{Conclusion And Future Work}\label{sec:conclusion}}

% \printbibliography

%%
%% The next two lines define the bibliography style to be used, and
%% the bibliography file.
\bibliographystyle{ACM-Reference-Format}
\bibliography{references-bibtex.bib}

\end{document}
\endinput
%%
%% End of file `sample-sigconf.tex'.
